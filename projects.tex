
\documentclass[11pt]{article}
\usepackage{common}
\usepackage{tikz}
\title{Final Projects  \\ CS 287: Statistical Natural Language Processing}
\date{}

\begin{document}

\maketitle{}
\section{Overview}

\begin{itemize}
\item Project Proposal; March 25th
\item Status Update; April 15th
\item Final Project Presentation; April 26th
\item Final Project Due; April 26th
\end{itemize}

\vspace{-1.75cm}
\section{Introduction}


\section{Bibliography}

A crucial element of the project will be reading and responding
intellectually to contemporary research in the field. Much of the
class has been devoted to developing vocabulary for reading papers on
these topics.  With this in mind, we will evaluate the project heavily
based on their processing of related work. Before jumping on a project
be sure to first find related references. Sometimes it can be hard to 
find \texttt{good} references on a topic (particularly these days). As 
a proxy  we recommend looking at papers from

NLP
\begin{itemize}
\item ACL
\item NAACL
\item EMNLP
\end{itemize}

Deep Learning/ Machine Learning
\begin{itemize}
\item ICML
\item NIPS
\item ICLR
\item AISTATS
\item UAI
\end{itemize}

Additionally, NLP has several shared tasks which contain data and evaluation 
on a specific problem: 

\begin{itemize}
\item SemEval
\item CoNLL Shared Tasks
\item KDDCup
\end{itemize}

It has also become common to post new papers on the Arxiv. Note though that these 
papers are often not peer reviewed. Be sure to be aware of this and keep it in mind.

\section{Process}

\subsection{Project Proposal}

The first deadline will be a project proposal. 
For the proposal we would like a one-page document describing 

\begin{itemize}
\item the motivation of the project
\item the research question being asked
\item the experimental setup
\item clear empirical metrics and baselines 
\item proposal for evaluating the success of the project

\item a bibliography of three (3) current references for the project   
\end{itemize}
 

The proposal will be graded on mainly on completeness. It will allow us to assign you a mentor 
and to propose early suggestions for related work and experiments.   

\subsection{Status Update}


\subsection{Presentation}


\subsection{Final Writeup}

For the final project, we will limit you to ten (10) pages of writing.
It should follow the Latex template that we have been using for the homework 
including background, main body, related work, experimental work, and discussion. 
Note that you should be judicious with your use of 10 pages. 

\section{}

\end{document}