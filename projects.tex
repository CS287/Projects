
\documentclass[11pt]{article}
\usepackage{common}
\usepackage{tikz}
\usepackage{hyperref}
\title{Final Projects  \\ CS 287: Statistical Natural Language Processing}
\date{}

\begin{document}

\maketitle{}

\section{Important Deadlines}

The final project will consist of the following four aspects:

\begin{enumerate}
\item Project Proposal
\item Status Update
\item Final Paper
\item Oral presentation in class
\end{enumerate}
\emph{No late days for any course project deadline.}

\vspace{0.25cm}

\noindent These deliverables will be due at the following deadlines:

\begin{center}
\begin{tabularx}{0.4\linewidth}{lr}
  \toprule
  Aspect & Deadline \\
  \midrule
  Project Proposal & 3/25, 5pm \\
  Status Update & 4/15, 5pm \\
  Oral Presentations&  4/26, in class \\
  Final Project &  5/1, 5pm \\
  \bottomrule
\end{tabularx}
\end{center}


\section{Goals and Scope}\label{goals-and-scope}

This course has been devoted to coverage of research
topics in modern natural language processing. The final 
project is the capstone of the course and a chance for you 
to engage with a difficult research problem in the area. 
Students are expected to design and carry out final
projects working in teams of two. You can use the Piazza forum
to find partners. 
As you have already implemented several research systems, 
it is no longer a stretch to build one of your own :).



% It is
% intended to encourage you to integrate ideas from different course
% components and allow you to delve deeper into areas of interest.
% Projects may be implementation, testing, and analysis of algorithms
% mentioned in lecture or described in the text but not covered in
% assignments (e.g., alternative search or planning algorithms) or design
% and implementation of an intelligent system that combines algorithms and
% representations from several course components (e.g., a more complex
% game-playing program or an intelligent advising system). Pick something
% that interests you



A crucial element of the project will be reading and responding
intellectually to contemporary research in the field. Much of the
class has been devoted to developing vocabulary for reading papers on
these topics.  With this in mind, we will evaluate the project heavily
based on their processing of related work. Before jumping on a project
be sure to first find related references. As 
a proxy we recommend looking at papers from:

\air 

\noindent Natural Language Processing
\begin{itemize}
\item ACL/NAACL/ EMNLP
\end{itemize}

\noindent  Deep Learning/ Machine Learning
\begin{itemize}
\item ICML/NIPS/ICLR/AISTATS/UAI
\end{itemize}

Additionally, NLP has several shared tasks which contain data and evaluation 
on a specific problem: 

\begin{itemize}
\item SemEval/CoNLL Shared Tasks/KDDCup
\end{itemize}

It has also become common to post new papers on the Arxiv. Note though that these 
papers are often not peer reviewed. Be sure to be aware of this and keep it in mind.

Plausible projects include: reimplementing and extending past papers, 
applying algorithms from the class to new domains, presenting comprehensive 
analysis and results to new language domains, and experimenting with 
new techniques on optimizing or building large-scale systems. 

If you are having trouble finding a topic yourself, a list of possible
course ideas is given on the\href{https://github.com/CS287/Projects/wiki}{projects wiki}. Their
content and scope are meant to be suggestive, not definitive. The
teaching staff would be delighted to talk with you about
possibilities. Each suggestion has an associated TF; he would be a
good person with whom to speak first, but feel free to contact any of
us. After the proposal, each final group will be assigned a course
mentor who will be the best point of contact throughout the project.

We will evaluate your project on the concepts it investigates and the
results and on how well it demonstrates your comprehension of the
concepts, techniques and issues we have covered in the class. The
project grade will incorporate evaluation of the proposal,
presentation (oral or poster) and final paper quality. As there is no
final exam for CS 287, your final project is the major integrative
element of coursework. It will account for 20\% of your overall course
grade.


\section{Deliverables}

\subsection{Project Proposal}

To ensure that you choose an appropriate project, you are required to
turn in a 1--2 page project proposal. The proposal should begin with a
clear, unambiguous statement of your topic, and include all of the
following:

\begin{itemize}
\item the motivation of the project
\item the research question being asked
\item the experimental setup
\item clear empirical metrics and baselines 
\item proposal for evaluating the success of the project

\item a bibliography of three (3) current references for the project   
\end{itemize}

The proposal will be graded on mainly on completeness. It will allow us to assign you a mentor 
and to propose early suggestions for related work and experiments.   

\subsection{Status Update}

To ensure that you are on track with the project and to identify any
issues on time, you are required to submit a short (3-page) report
with a status update. The report should describe the problem you are
working on, the progress you've made so far, and any problems you came
across that you would like to get help with. 

\subsection{Presentation}

The oral presentations will be on April 26th. These presentations are
a chance to explain your problem and approach, showcase what you've
accomplished, and get advice on surmounting any hurdles you've
encountered. Students are expected to attend both presentation
sessions, as they provide an opportunity for you to learn from each
other.


Oral presentations will be allocated 10 minutes, and should be
focused on key issues. You are encouraged to bring less than 4
slides, because often just one diagram or chart can explain the
essence of your idea and save lots of presentation time. You need
not prepare fancy graphics; just come prepared to explain your topic
and share what you have discovered.


\subsection{Report}

Finally, you must submit a written report on your project and the complete,
well-documented source code for it. The report should be a maximum 10
pages in length (this is actually quite short, so you should be judicious 
with your details and graphics).

The report format should be exactly the same as the homework
assignments for the class and use the same Latex template. Your 
goal should not be to innovate in the format, but to use the 
standard we have had in class, and present details in a clear 
and, hopefully, obvious way. 


\end{document}